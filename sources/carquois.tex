\chapter{Carquois et représentations}

% \begin{df}
% Un carquois \(Q\) est une couple \((Q_0, Q_1, s, t)\) 
% où \(Q_0\) est un ensemble fini dont les points son appelés sommets
% et \(Q_1\) est un ensemble fini dont les points son appelés fléches.
% Et \(s,t\) deux  fonctions \(s\colon Q_1\to Q_0\),
% appelés source and target.
% \end{df}

% We dont care about the names of the points,
% so we draw points.
% \begin{ex}
% \[\begin{tikzcd}[cells={nodes={inner sep=5pt}}]
% 	& \circ \\
% 	\circ && \circ
% 	\arrow["\alpha"', from=1-2, to=2-1]
% 	\arrow["\beta", from=1-2, to=2-3]
% 	\arrow["\gamma"', shift right, from=2-3, to=2-1]
% 	\arrow["\delta", shift left, from=2-3, to=2-1]
% \end{tikzcd}\]

% other 
% \[
% \begin{tikzcd}[cells={nodes={inner sep=5pt}}]
%   & 1 \\
%   2 && 3
%   \arrow["\alpha"', from=1-2, to=2-1]
%   \arrow["\beta",  from=1-2, to=2-3]
%   \arrow["\gamma"', shift right, from=2-3, to=2-1]
%   \arrow["\delta",  shift left,  from=2-3, to=2-1]
% \end{tikzcd}
% \]

% \end{ex}


% \begin{ex}
% \[\begin{tikzcd}
% 	0 & 1 & 2 & \cdots & n
% 	\arrow[from=1-1, to=1-2]
% 	\arrow[from=1-2, to=1-3]
% 	\arrow[from=1-3, to=1-4]
% 	\arrow[from=1-4, to=1-5]
% \end{tikzcd}\]
% \end{ex}


% \begin{df}
% Une répresentation d'un carquois \(Q\) est la donnée
% des eapces vectoriels \(V_x\) pour chaque \(x\in Q_0\)
% et des fonctions linéaires \(V_\alpha\colon V_x\to V_y\)
% pour chaque \(\alpha\in Q_1\).
% \end{df}

% Essentialement on rémplace les fléches par des applications linéaires.

% \red{i like the arrows of the yellow book of rep theory of assem, see eg p. 2}
% \red{maybe i can use that book as refernce}

% \(A \xrightarrow{f} B\)


hellop.
Une fonction \(\varphi\colon A\to B\) et une autre \(\varphi\colon A\farrow B\) 


\[
x\to x b\quad  x\farrow x b
\]