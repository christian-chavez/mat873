
\section{Groupe d'homologie}
\red{Lecture 2026-01-22}
see my notes,
check Elementary Applied topology, Ghrist, page 30


Tout au long de cette section, \(\mathbb{F}\) désigne un corps.
On travaille avec \(\mathbb{F} = \Z_2 \).

\begin{df}
Une complexe de chaîne 
est une paire  \(\left( C_\bullet, \partial_\bullet \right)\),
où \(C_\bullet = (C_i)_{i\in \Z}\) est une famille de espaces vectoriels sur \(\F\)
et \(\partial_\bullet = (\partial_i\colon C_i\to C_{i-1})_{i\in \Z}\) est une famille \red{:/}  de applications linéaires
\[
\cdots \to C_i \xrightarrow{\partial_i} C_{i-1} \xrightarrow{\partial_{i-1}} C_{i-2} \to \cdots
\]
tel que 
\[
\partial_{i-1}\circ \partial_{i} = 0
\]
pour chaque \(i\in\Z\).
On dit que l'espace \(C_i\)  est de degré \(i\)
et que \(\partial_i \) est le differetiel \red{confirm this terminology} 
de degré \(i\).
\end{df}

Comme d'habitude, on omit the endowing stuff, et on denote le complexe de chaîne
simplement par \(C_\bullet\),
or even par \(C\) seulement.

\begin{rmk}\hfill
\begin{itemize}
\item On se souvient toujours que le dubindex de \(\partial\) est le index de son domaine.
\item On remarque que la condition \(\partial_{i-1}\circ \partial_{i}\) est équivalente à \(\im \partial_i \subseteq \Ker \partial_{i-1} \).
\item On note \(B_i(C_\bullet) = \im \partial_i\) et \(Z_i(C_\bullet) = \Ker \partial_i\) pour chaque \(i\in \Z\).
\item On oublie le subindex de \(\partial_i\). Donc on écrit \(\partial^2 = 0\).
\end{itemize}

\end{rmk}

\red{quelle est la difference avec complexe differentiel?}


\begin{df}
Le i-ème groupe d'homologie d'un complexe simplicial \(\left( C_\bullet, \partial_\bullet \right)\) est 
\[
H_i(C_\bullet) = \Ker \partial_i / \im \partial_{i+1}.
\]
\end{df}



On remarque que \(H_i\) est un espace vectoriel,
éntant que qoutient des espaces vetoriels.

Avec notre notation de antes, 
on a \(H_i(C_\bullet) = Z_i/ B_i\).

\subsection{``but d'ajourdhui'}

on a vu les définitions précedentes sans réference a ce on avait étudie avant.
Maintenant, on fait la connection.
Étant donné un complexe simplicial abstrait \(A\),
on voudrait  construire un complexe de chaînes 
\(C(A)\).
même si  on donne un complexe simplices geométrique.


Soit \(A \) un complexe simplicial abstrait.
Pour chaque \(i\in \Z\),
on pose 
\(C_i\) l'espace vectoriel d ebase l'ensemble de i-simplexes dans \(A\). (i.e. de dim i)

Alors 
\begin{align*}
C_0 &= \langle\,  \mathrm{points} \, \rangle\\
C_1 &= \langle\,  \text{arêtes} \, \rangle\\
C_2 &= \langle\,  \mathrm{triangles} \, \rangle\\
    &\ \, \vdots                           \\
C_i &= \langle\,  i\mathrm{-simplexes} \, \rangle
\end{align*}

On remarque que \(C_i = 0\), l'espace vectoriel nul, pour Tout \(i \in \Z_{\leq 1}\).
Alors on pose \(C(A) = (C_i)_{i\in\Z}\).

On construit Maintenant les applications linéaires.

\[
\partial_1 \colon C_1 \to C_0, \quad ab\mapsto a+b
\]

% \(\cdot - \cdot\mapsto \) 
\red{définir \(\partial_i\)}

Si \(A\) est un complexe simplicial abstrait,
on a le \(i\)-ème groupe d'homologie 
\[
H_i(A) = H_i(C(A)).
\]
on note \(\beta_i = \dim H_i(A)\).
La dimension étant que espace vectoriel,
laquel est égal a dim ker - dim img, 
par un théorème d'algèbre linéaire.


\begin{ex}
Soit 
\[
A = \left\{ 
        [5,6], [1,2,3], [1,4], [3,4]
 \right\}   
\]

\begin{figure}[H]
    \centering
    \incsvg{1}{cs-un-dim-trou-un-arete}
    %\caption{}
    %\label{fig:cs-un-dim-trou-un-arete}
\end{figure}
%
\noindent On va (i) construire un complexe de chaîne et (ii) calculer le groupe de homologie à chaque point\red{?}

\[
\adjustbox{scale=0.85}{%
\begin{tikzcd}[
    column sep={1.4em},
    % 1. Force the first column to align to the West (Left)
    /tikz/column 1/.append style={anchor=base west},
    % 2. Draw the backgrounds behind the diagram
    execute at end picture={
        \begin{pgfonlayer}{background}
            % Gray box for the top row (from alias 'r1start' to 'r1end')
            \node[fit=(r1start) (r1end), fill=gray!10, inner sep=2.5pt] {};
            % Gray box for the bottom row (from alias 'r4start' to 'r4end')
            \node[fit=(r4start) (r4end), fill=gray!10, inner sep=2.5pt] {};
        \end{pgfonlayer}
    }
]
% Row 1: Add aliases to start and end
|[alias=r1start]| \mathrm{type} & & \text{tétraèdres} & \mathrm{triangles} & \text{arêtes} & \mathrm{sommets} & |[alias=r1end]| \mathrm{nul} \\
\mathrm{complexe} & \cdots \arrow[r] & C_3 \arrow[r, "\partial_3"] & C_2 \arrow[r, "\partial_2"] & C_1 \arrow[r, "\partial_1"] & C_0 \arrow[r, "\partial_0"] & 0 \\
\mathrm{base} & \cdots \arrow[r] & 0 \arrow[r, "\partial_3"] & \langle 123\rangle \arrow[r, "\partial_2"] & {\langle 12,13,23,34,14,56\rangle} \arrow[r, "\partial_1"] & {\langle 1,2,3,4,5,6\rangle} \arrow[r, "\partial_0"] & 0 \\
% Row 4: Add aliases to start and end
|[alias=r4start]| \dim(C_i) & & 0 & 1 & 6 & 6 & |[alias=r4end]| 0
\end{tikzcd}
}
\]

D'abord on trouve les applications \(\partial_i\).
Comme notres espaces vectoriels ont dimension finie,
on \red{regard our linear maps as matrices}.
On remarque que \(\partial_3 = 0\) et \(\partial_0\) car 
\(\mathrm{dom}(\partial_3) = 0\) 
et 
\(\mathrm{dom}(\partial_0) = 0\),
respectivement.
Pour trouver \(\partial_1\) et \(\partial_2\),
we look at what they do to the bases.
Pour \(\partial_1\), on a 
\begin{align*}
\partial_1(12) &= 1+2 = \textcolor{gray}{1}\cdot 1 + \textcolor{gray}{1}\cdot 2 + \textcolor{gray}{0} \cdot 3 + \textcolor{gray}{0}\cdot 4 + \textcolor{gray}{0} \cdot 5 + \textcolor{gray}{0} \cdot 6, \\
\partial_1(13) &= 1+3 = \textcolor{gray}{1}\cdot 1 + \textcolor{gray}{0}\cdot 2 + \textcolor{gray}{1} \cdot 3 + \textcolor{gray}{0}\cdot 4 + \textcolor{gray}{0} \cdot 5 + \textcolor{gray}{0} \cdot 6 ,\\
\partial_1(23) &= 2+3 = \textcolor{gray}{0}\cdot 1 + \textcolor{gray}{1}\cdot 2 + \textcolor{gray}{1} \cdot 3 + \textcolor{gray}{0}\cdot 4 + \textcolor{gray}{0} \cdot 5 + \textcolor{gray}{0} \cdot 6 ,\\
\partial_1(34) &= 3+4 = \textcolor{gray}{0}\cdot 1 + \textcolor{gray}{0}\cdot 2 + \textcolor{gray}{1} \cdot 3 + \textcolor{gray}{1}\cdot 4 + \textcolor{gray}{0} \cdot 5 + \textcolor{gray}{0} \cdot 6 ,\\
\partial_1(14) &= 1+4 = \textcolor{gray}{1}\cdot 1 + \textcolor{gray}{0}\cdot 2 + \textcolor{gray}{0} \cdot 3 + \textcolor{gray}{1}\cdot 4 + \textcolor{gray}{0} \cdot 5 + \textcolor{gray}{0} \cdot 6 ,\\
\partial_1(56) &= 5+6 = \textcolor{gray}{0}\cdot 1 + \textcolor{gray}{0}\cdot 2 + \textcolor{gray}{0} \cdot 3 + \textcolor{gray}{0}\cdot 4 + \textcolor{gray}{1} \cdot 5 + \textcolor{gray}{1} \cdot 6 .
\end{align*}
La matrice associée est donc
\[
M_{1} = 
\sbordermatrix{
 & \textcolor{gray}{12} & \textcolor{gray}{13} & \textcolor{gray}{23} & \textcolor{gray}{34} & \textcolor{gray}{14} & \textcolor{gray}{56} \cr
\textcolor{gray}{1 }& 1 & 1 & 0 & 0 & 1 & 0 \cr
\textcolor{gray}{2 }& 1 & 0 & 1 & 0 & 0 & 0 \cr
\textcolor{gray}{3 }& 0 & 1 & 1 & 1 & 0 & 0 \cr
\textcolor{gray}{4 }& 0 & 0 & 0 & 1 & 1 & 0 \cr
\textcolor{gray}{5 }& 0 & 0 & 0 & 0 & 0 & 1 \cr
\textcolor{gray}{6 }& 0 & 0 & 0 & 0 & 0 & 1
}.
\]
Ceci est d'accord avec l'info du diagramme ci-dessus,
ou se nos indica que \(\partial_1\) est une application linéaire entre espaces de dimension \(6\).
Pour \(\partial_2\), on a 
\begin{align*}
\partial_2(123) &= 12 + 13 + 23\\
                &= \textcolor{gray}{1}\cdot 12 + \textcolor{gray}{1}\cdot 13 + \textcolor{gray}{1} \cdot 23 + \textcolor{gray}{0}\cdot 34 + \textcolor{gray}{0} \cdot 14 + \textcolor{gray}{0} \cdot 56
\end{align*}
et sa matrice associée est 
\[
M_{2} = 
\sbordermatrix{
 & \textcolor{gray}{123}  \cr
\textcolor{gray}{12 } & 1 \cr
\textcolor{gray}{13 } & 1 \cr
\textcolor{gray}{23 } & 1 \cr
\textcolor{gray}{34 } & 0 \cr
\textcolor{gray}{14 } & 0 \cr
\textcolor{gray}{56 } & 0
}
\]

On a placé les nombre en gray seulement pour aide pedagogique.
On vérifié que \(\partial_{i-1}\circ \partial_i = 0\)
pour chaque \(i\in \Z\).
Si \(i\geq 3\) ou \(i\leq 1\), alors \(\partial_i = 0\), et donc 
\(\partial_{i-1}\circ \partial_i = 0\).
Alors il reste seulement à vérifier pour \(i = 2\).
On a 
\begin{align*}
M_1 M_2 &=
\left[\begin{smallmatrix}
    \vrule width 0pt height 6pt
1 & 1 & 0 & 0 & 1 & 0 \\
1 & 0 & 1 & 0 & 0 & 0 \\
0 & 1 & 1 & 1 & 0 & 0 \\
0 & 0 & 0 & 1 & 1 & 0 \\
0 & 0 & 0 & 0 & 0 & 1 \\
0 & 0 & 0 & 0 & 0 & 1
    \vrule width 0pt depth 2pt
\end{smallmatrix}\right]
\left[\begin{smallmatrix}
\vrule width 0pt height 6pt
1 \\ 1 \\ 1 \\ 0 \\ 0 \\ 0
\vrule width 0pt depth 2pt
\end{smallmatrix}\right] \\
&= 1 \cdot \!\left[\begin{smallmatrix} 
\vrule width 0pt height 6pt 
1 \\ 1 \\ 0 \\ 0 \\ 0 \\ 0 
\vrule width 0pt depth 2pt
\end{smallmatrix}\right]
+ 1 \cdot \!\left[\begin{smallmatrix}
\vrule width 0pt height 6pt 1 \\ 0 \\ 1 \\ 0 \\ 0 \\ 0 \vrule width 0pt depth 2pt
\end{smallmatrix}\right]
+ 1 \cdot \!\left[\begin{smallmatrix}
\vrule width 0pt height 6pt 0 \\ 1 \\ 1 \\ 0 \\ 0 \\ 0 \vrule width 0pt depth 2pt
\end{smallmatrix}\right]
+ 0 \cdot \!\left[\begin{smallmatrix}
\vrule width 0pt height 6pt 0 \\ 0 \\ 1 \\ 1 \\ 0 \\ 0 \vrule width 0pt depth 2pt
\end{smallmatrix}\right]
+ 0 \cdot \!\left[\begin{smallmatrix}
\vrule width 0pt height 6pt 1 \\ 0 \\ 0 \\ 1 \\ 0 \\ 0 \vrule width 0pt depth 2pt
\end{smallmatrix}\right]
+ 0 \cdot \!\left[\begin{smallmatrix}
\vrule width 0pt height 6pt 0 \\ 0 \\ 0 \\ 0 \\ 1 \\ 1 \vrule width 0pt depth 2pt
\end{smallmatrix}\right] \\
&= \left[\begin{smallmatrix}
\vrule width 0pt height 6pt 1+1+0 \\ 1+0+1 \\ 0+1+1 \\ 0 \\ 0 \\ 0 \end{smallmatrix}\right]
= \vrule width 0pt depth 2pt
\left[\begin{smallmatrix}
\vrule width 0pt height 6pt 2 \\ 2 \\ 2 \\ 0 \\ 0 \\ 0 \vrule width 0pt depth 2pt
\end{smallmatrix}\right]
= \left[\begin{smallmatrix}
\vrule width 0pt height 6pt 0 \\ 0 \\ 0 \\ 0 \\ 0 \\ 0 \vrule width 0pt depth 2pt
\end{smallmatrix}\right].
\end{align*}
\end{ex}
Il suit que \(\partial_1\circ \partial_2 = 0\).
On a bien montré que c'est un complexe de chaîne.

On calcule maintenant les groupes d'homologie.
Pour \(i\geq 3\),  le domain de \(\partial_i \colon 0\to C_{i-1}\) 
est \(0\), son noyau est donc \(0\).
Il suit que 
\[
H_i 
= \frac{\Ker \p_i }{\im \p_{i+1}}
= \frac{0}{\im \p_{i+1}}
= 0.
\]
Ceci refleja el hecho de que no hay huecos de dimensiones 3 o mayor.
Les groupes d'homologie qui nous interesent son :
\[
H_2 = \frac{\Ker \p_2 }{\im \p_{3}},
\quad
H_1 = \frac{\Ker \p_1 }{\im \p_{2}},
\quad
H_0 = \frac{\Ker \p_0 }{\im \p_{1}}.
\]
C'est très fácile à voir que 
\(\im \p_{3} = 0\) et que 
\(\Ker \p_0 = C_0\). \red{should elaborate on this?}
De plus, comme \(\p_2\) est injetif, \(\Ker \p_2 = 0\).
Alors 
\[
H_2 
% = \frac{\Ker \p_2 }{\im \p_{3}}
= \frac{0}{\im \p_{3}}
= 0.
\]
Pour trouver \(\Ker \p_1\),
on trouve une base de l'espace colonne 
de \(M_1\),
ce que est un task de algèbre linéaire.

\begin{align*}
% 1. Start: Top Right Matrix
\left[
\begin{array}{ccc|ccc}
1 & 1 & 0 & 0 & 1 & 0 \\
1 & 0 & 1 & 0 & 0 & 0 \\
0 & 1 & 1 & 1 & 0 & 0 \\\hline
0 & 0 & 0 & 1 & 1 & 0 \\
0 & 0 & 0 & 0 & 0 & 1 \\
0 & 0 & 0 & 0 & 0 & 1
\end{array}
\right]
% Arrow down: R2+R1 and R6+R5
&\xrightarrow{\substack{R_2 + R_1 \\ R_6 + R_5}}
\left[
\begin{array}{ccc|ccc}
1 & 1 & 0 & 0 & 1 & 0 \\
0 & 1 & 1 & 0 & 1 & 0 \\
0 & 1 & 1 & 1 & 0 & 0 \\
0 & 0 & 0 & 1 & 1 & 0 \\
0 & 0 & 0 & 0 & 0 & 1 \\
0 & 0 & 0 & 0 & 0 & 0
\end{array}
\right] \\[2em]
% Arrow left: R3+R2
&\xrightarrow{R_3 + R_2}
\left[
\begin{array}{ccc|ccc}
1 & 1 & 0 & 0 & 1 & 0 \\
0 & 1 & 1 & 0 & 1 & 0 \\
0 & 0 & 0 & 1 & 1 & 0 \\
0 & 0 & 0 & 1 & 1 & 0 \\
0 & 0 & 0 & 0 & 0 & 1 \\
0 & 0 & 0 & 0 & 0 & 0
\end{array}
\right] \\[2em]
% Arrow down: R4+R3
&\xrightarrow{R_4 + R_3}
\left[
\begin{array}{ccc|ccc}
1 & 1 & 0 & 0 & 1 & 0 \\
0 & 1 & 1 & 0 & 1 & 0 \\
0 & 0 & 0 & 1 & 1 & 0 \\
0 & 0 & 0 & 0 & 0 & 0 \\
0 & 0 & 0 & 0 & 0 & 1 \\
0 & 0 & 0 & 0 & 0 & 0
\end{array}
\right]
\end{align*}


\subsection{Complexe de Cech}

\section{2026-01-29}

\begin{df}
Une suite exacte corte d'espaces vectoriels
est 
\end{df}

on peut le voir comme un cas particulier d'un complexe de chaine
ou \(C_i = 0\) pour tou \(i<0, i>2\).

\red{convention: when we say the diagram commutes, we mean the solid lines. the dashed arros are about the claim of the thm}

\begin{lm}
Donné un diagramme commutatif de lignes exactes 
\[
\begin{tikzcd}
0 \arrow[r] & X \arrow[d, "u"] \arrow[r, "f"] & Y \arrow[r, "g"] \arrow[d, "v"] & Z \arrow[d, "w", dashed] \arrow[r] & 0 \\
0 \arrow[r] & X' \arrow[r, "f'"]              & Y' \arrow[r, "g'"]              & Z' \arrow[r]                       & 0
\end{tikzcd}
\]
Il existe une fonction linéaire \(w\colon Z\to Z' \)
tel que le carré à droite commute,
i.e. \(w\circ g = g' \circ v\).
\end{lm}

\red{maybe the zeros to the left are not needed.}

\begin{proof}
Soit \(z\in Z\).
\end{proof}


\begin{df}
Une filtration d'un complexe simplicial \(K\) 
est une suite de complexes simpliciaux
\[
K(0)\subseteq K(1) \subseteq \cdots \subseteq K(n) = K
\] 
\end{df}


\red{exemple: the big grid}