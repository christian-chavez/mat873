\chapter{Des choses de base}


\red{Lecture 2026-01-08}

may be check Schenck Chapter 4
Homology I: Simplicial Complexes
to Sensor Networks

and 
chap 3 of 
Edelsbrunner

\section{Motivation}

Soit   \(K\) un espace topologique.

\subsection{Les groupes d'homologie}

On voudrait compter les trous d'un espace topologique quelconque.
Par exemple, 

\begin{itemize}
\item la ligne réel n'a pas de trous
\item si on prendre un segment de recte et join the bouts, on obtient un circle. il n'y a pas de trou dans le borde (otherwise it is not a circle). mais il enclose un trou.
\item 
\item la sphère n'a pas de trous dans la surface (otgerwise is not an sphère), mais elle a un trou dedans, mais ce different de le trou que le circle enclose
\item la sphère moins un disque a deux trous. pas vraiment. 
\item la figure du \texttt{8} a deux trous.i
\end{itemize}


\begin{figure}[H]
    \centering
    \incsvg{1}{efwjbjkwf}\quad
    \incsvg{1}{sphere}\quad
    \incsvg{1}{sphere-trou}
    %\caption{}
    %\label{fig:efwjbjkwf}
\end{figure}

\(K_1\) est la ligne seulement, pas l'air qu'elle contient.
Sino, on aurait peint l'interiéur.


Pour illustrer, on note par 
\(H_0\) le nombre de composantes conexes,
\(H_1\) le nombre de trous de dimension 1, i.e., trous planars,
\(H_2\) le nombre de trous bidimensionals, etc.

Alors dans les figures ci-dessus, on a 
\begin{alignat*}{3}
H_0(K_1) &= \Z,\quad  && H_1(K_1) = \Z^4,\quad  && H_2(K_1) = 0,\\
H_0(K_2) &= \Z,\quad  && H_1(K_2) = \Z,\quad  && H_2(K_2) =\Z,\\
H_0(K_3) &= \Z,\quad  && H_1(K_3) = 0,\quad  && H_2(K_3) = 0.
\end{alignat*}

We ``encode'' the number of conected components in the exponent.
Notice that we always have \(H_0\geq 1\).


On va voir deux théorèmes importants :

\begin{thm}
Soient \(K\) et \(K'\) deux espaces topologiques.
Si \(K\cong K'\), 
alors \(H_p(K)\cong H_p(K')\).
\end{thm}

On remarque que le premier \(\cong\) est un isomorphisme d'espaces topologiques 
et le deuxième \(\cong\) est un isomorphisme de groupes.


\begin{thm}
Tout espace topologique est isomorphe à un complexe simplicial.
\red{est-ce que ceci est vrai?}
\end{thm}

\section{Complexes simpliciaux}

BUT : Définir \emph{complexe simplicial}.


\begin{df}
Un \(n\)-simplex \(\sigma\) est 
\end{df}

\begin{df}
On dit que \(n\) points \(u_1, \dots, u_n\in \R^d\)
sont \textbf{linéairement independants}
si 
$$
\dim \spn \left\{ u_1, \dots, u_n \right\} = n
$$
\end{df}
 

Il n'est pas necesaire d'ajouter la condition \(n\leqslant d\)
dans la définition.

% \red{Par définition, un singleton n'est jamais affinement independant 
% car l'ensemble \(\{0\}\) n'est pas linéairement independant. THIS IS FALSE.}
% \red{the def says that we should consider the n points after removing the one that we sustracted.
% Thus every singleton is affinement independant car the empty set is l.i}
\red{ 
* Affine Independence is interesting from R² onwards, not R¹. Nest pas?
is it worth mentioning sth like this?
}

\begin{df}
On dit que \(n+1\) points \(u_0, \dots, u_n\in \R^d\)
sont \textbf{affinement independants}
si les \(n\) points \(u_0 - u_1, \dots, u_0 - u_n\)
sont linéairement independants
\end{df}

Le point \(u_0\) n'est pas especial.
On peut chosir un autre point de notre liste, disons \(u_j\),
et verifier que 
\(u_j - u_1, \dots, u_j - u_n\)
est linéairement independant.

Pour illustrer l'idée d'être affinement indépendants,
on considere trois points.
Si on prend \(u_0\) comme origen,
alors on a deux vecteurs, et on checke si son colineaires ou pas.
les points aff. ind. si il ne sont pas dans la même 
ligne.
l'origin n'importe pas, 
il s'agit d'une question geometrique 
et pas de la position dans le plan.

\begin{figure}[H]
    \centering
    \incsvg{1}{idde-affinement-indep}
    %\caption{}
    %\label{fig:idde-affinement-indep}
\end{figure}

\begin{ex}
\begin{enumerate}[label=(\roman*)]
\item Tout ensemble linéairement independant est affinement independant, mais pas au contraire.
\item Tout ensemble avec un point est affinement independant. En particulier, l'esemble \(\{0\}\), qui est linéairement dépendant,  est affinement independant. Ceci suit du fait que \(\varnothing\) est linéairement independant. Peut le lecteur montrer que tout ensemble avec deux points differents est affinement independant?
\item Dans \(\R^n\), l'ensemble \(\{0, \mathbf e_1, \dots, \mathbf{e}_n\}\) est affinement
independant.
Ici \(\mathbf e_i\) denote le \(i\)-ème vecteur de la base canonique.
Au fait, l'ensemble \(\{0, \mathbf e_1, \dots, \mathbf{e}_i\}\) est affinement
independant, pour chaque \(i\geq 1\).
\end{enumerate}
\end{ex}
 

\begin{cis}
Soit \(\{u_0, \dots, u_n\}\) linéairement independant.

\begin{enumerate}[label=(\alph*)]
\item Quel point peut-on l'ajouter so that 
on obtient un esemble affinement independant ?
\item Montrer que, pour chaque \(j\), l'ensemble \(X=\{u_0-u_j, \dots, u_n-u_j\}\) n'est jamais  affinement independant. Cependant, \(X\setminus \{0\}\) est toujours affinement independant.
\item Soit \(v_k = u_0 + \cdots + u_k\).
Montrer que \(\{v_1, v_2, \dots, v_k\}\) est affinement independant.
Pour quelles valeures de \(a\) est \(\{av_0,v_1, v_2, \dots, v_k\}\)
affinement independant ?
\end{enumerate}


\end{cis}


La combinaison convexe de   \(u_1, \dots, u_n\in \R^d\) (affinement independants ou pas)
est 
\red{give intuition}
\[
\conv\{u_0, \dots, u_n\}
=
\left\{ 
\sum_{i=0}^{n}\lambda_iu_i \ \bigg|\  \lambda_i\geq 0 \text{ et } \lambda_1+ \cdots + \lambda_n =1
\right\}
\]

\begin{df}
Un  \(n\)-\textbf{simplex}
est la combinaison convexe de d'un ensemble fini de points affinement independants.
La dimension d'un \(n\)-simplex \(\sigma = \conv\{u_0, \dots, u_n\}\)
est  \(\dim \sigma = n\).
\end{df}

\begin{rmk}
deux simplex peuvent avoir la même dimension 
même si apartient a differents espaces.
E.g. un triangle dans le plan, un triangle dans l'espace, dans R4
il a toujours dimension 3.
\end{rmk}

\red{define standard simplexes?}
there are many segments, triangles and tetahedra, mais 
il y quelconques que sont distinguished.

\begin{ex}
\begin{enumerate}[label=(\roman*)]
\item Pour \(n=0\), on a \(\sigma = \conv\{u_0\} = \{u_0\}\), un singleton.
\item segment
\item triangle
\item tetraèdre
\end{enumerate}
\end{ex}

Soit \(\sigma = \conv\{ u_0, \dots, u_n \}\)
un \(n\)-simplex 
et 
\(S\subset \{ u_0, \dots, u_n \}\) une partie non vide.
Alors \(\tau = \conv S\) est un simplex de dimension \(|S| - 1 \).
On dit que \(\tau\) est une face de \(\sigma\).
On écrit \(\tau\leq \sigma\).

\begin{rmk}
On dit  que \(\tau\) est \textbf{propre} si \(\tau\neq \sigma\).
Dans ce cas là on note \(\tau < \sigma\).
\end{rmk}

\begin{ex}
Soient \(u_0, \dots, u_3\) affinement independants.
Alors \(\sigma = \conv\{u_0, \dots, u_4\}\)
est un tetraèdre
et \(\sigma = \conv\{u_0, \dots, u_3\}\) 
est un triangle.
\red{put a picture of a tetraèdre with a face shaded of some color}
en fait, est une face propre.
\end{ex}

Pour un simplexe \(\sigma=\conv\{u_0, \dots, u_n\}\),
l'envelope de \(\sigma\)
est 
\[
\bd \sigma = \bigcup_{\mathrm{face\ propre\ }\tau\leq \sigma} \tau.
\]
Ce signifie que l'envelope est la ligne exterieur, the boundary.
L'interiéur de \(\sigma\) est \(\sigma\setminus \bd \sigma\).
Un point appartient à l'interieur si tous les \(\lambda_i\) sont positifs.

On arrive à notre but de cette section.

\begin{df}
Un \textbf{complexe simplicial} 
\(K\) est un ensemble fini de 
simplexes tels que 
\red{but what if they come from different ambient spaces? 
we should mention this!}
\begin{enumerate}[label=(\roman*)]
\item si \(\sigma\in K\) et \(\tau\leq \sigma\), alors \(\tau\in K\) ;
\item si \(\sigma,\sigma'\in K\) et \(\sigma\cap \sigma'\neq \varnothing\), alors \(\sigma\cap \sigma' \leq \sigma  \).
\end{enumerate}
La \textbf{dimension} de \(K\) est \[
\dim K = \max_{\sigma\in K} \left( \dim \sigma \right).
\]
L'espace associé \red{underlying space} est 
\[
|K| = \bigcup_{\sigma\in K} \sigma.
\]
C'est un espace topologique avec la topologie
\red{inherited from the ambient Euclidean space in which the simplices live}
so what does this mean?
\end{df}

\red{i should put a whole page of examples and nonexamples. just drawings.}

\begin{ex}
On considère le complexe simplicial
\[
K= \left\{ 
    \conv\{u_0, u_1,u_2\}, 
    \conv\{u_0, u_1\},  
    \conv\{u_0, u_2\},  
    \conv\{u_1, u_2\}, 
    \conv\{u_2, u_3\}, 
    u_0, u_1,u_2, u_3.
 \right\}
\]
\red{shouldn't be the singletons instead of the actual points in there?}
Graphiquement : 
\begin{figure}[H]
    \centering
    \incsvg{1}{ex-simplex-tri}\\
    %\caption{}
    %\label{fig:ex-simplex-tri}
\end{figure}
\end{ex}

\red{pintar los dibujos le quita formalidad?}

\begin{ex}
Soit \(K\) le complexe simplicial du dernier exemple.
Si on ajoute le segment \(\conv\{u_1,u_3\}\),
on obtient le complexe suivant
\begin{figure}[H]
    \centering
    \incsvg{1}{ex-simplex-quat}\\
    %\caption{}
    %\label{fig:ex-simplex-quat}
\end{figure}
\end{ex}

et on peut avoir des choses qui ne sont pas connectées.

\begin{ex}[Non exemple]
Le problème ce que l'intersection 
des segment 
\(\conv\{u_1,  u_2\}\)
et 
\(\conv\{u_3,  u_4\}\)
n'est pas un element du \(K\).
Mais on peut bien l'ajouter.
\begin{figure}[H]
    \centering
    \incsvg{1}{ex-non-complexe-simplicial}\\
    %\caption{}
    %\label{fig:ex-non-complexe-simplicial}
\end{figure}
\end{ex}

\red{i like this quote from Edelsbrunner}
\red{It is often easier to construct a complex ab-
stractly and worry about how to put it into Euclidean space later, if at all.}


\begin{df}
\red{use the power set of the power set ? }
Un \textbf{complexe simplicial abstrait} \(A\)
est un ensemble fini des ensembles finis, sauf \(\varnothing\), 
tels que 
si \(\sigma\in A\) et \(\tau\subset \sigma\),
alors 
\(\tau\in A\). \red{this means \(A\) is closed below, sth like that}
La dimension de \(A\) est 
\[
\dim A = \max_{\sigma\in A} \left\{ |\sigma| - 1 \right\}.
\]
\red{explain the -1.}
\end{df}
\red{but then we have a new definition of simplex? DEFINE IT}

\begin{ex}
Un graphe est un example d'un complexe simplicial abstrait, 
en particulier un quiver, un arbre, hasse diagram,  etc.
\end{ex}

\red{i got confused while trying to solve Benv ex 1.1.4.i. 
There, s is a simplex in the combinatorial sense. Furthermore,
a simplicial complex is closed below by containments, ie it must contain the lower-dimensional elements of its simplices. however, simplices do not have that property. so \(\{1,2\}\) a perfect example of a simplex, but \(\{\{1,2\}\}\) is not a valid complex }

\begin{ex}
L'ensemble
\[
A = 
\left\{ 
    \left\{ 0,1,2 \right\},
    \left\{ 0,1 \right\},
    \left\{ 0,2 \right\},
    \left\{ 1,2 \right\},
    \left\{ 2,3 \right\},
    \left\{ 0 \right\},
    \left\{ 1 \right\},
    \left\{ 2 \right\},
    \left\{ 3 \right\}
\right\}
\]
\end{ex}

Pour un ensemble fini de points \(A = \left\{ a_0, \dots, a_n \right\}\),
on note 
par \([a_0, \dots, a_n]\)
le ensemble de tous parties 
nonvides de \(A\).
Équivalentement, \red{?}
\[
[a_0, \dots, a_n] = \mathscr{P}(\left\{ a_0, \dots, a_n \right\}) \setminus \left\{ \varnothing \right\}.
\]
\red{Benv notes: they omit the parethenses and commas altogether ``Lorsqu’il n’y a pas de confusion possible, on omet les parenthèses et les virgules pour les sim-
plexes, par exemple le simplexe {1, 2, 3} est simplement noté 123''}
\red{are there any books that use this convention?}

.

\red{i dont like this Convention from Benv:
Lorsqu’on note un complexe simplicial,
on se permet d’omettre de mentionner la présence de certains simplexes lorsque leur présence est assurée par
celle d’un autre simplexe dont c’est la face. Par exemple, si on déclare que C = {1234, 45}, ça signifie que
C = {1234} ∪ {234, 134, 124, 123} ∪ {12, 13, 14, 23, 24, 34, 45} ∪ {1, 2, 3, 4, 5}.
%
i think it is better to stick to the parenthesis. does any author use this? can this convention be improved? maybe say that 123 denotes the powerset of \{123\} without \(\varnothing\)
}

\begin{ex}
On a 
\[
[0,1,2] = 
\left\{ 
    \left\{ 0,1,2 \right\},
    \left\{ 0,1 \right\},
    \left\{ 0,2 \right\},
    \left\{ 1,2 \right\},
    \left\{ 0 \right\},
    \left\{ 1 \right\},
    \left\{ 2 \right\},
    \left\{ 3 \right\}
\right\}.
\]
\red{importa el orden? para qué? say \([0,1,2] = [2,1,0]\) ?}
\end{ex}

\red{poner un ejemplo que illustre que las etiquetas no son importants}


\begin{rmk}
Si \(K\) est un complexe simplicial geométrique,
son complexe abstrait \(A(K)\) est donné par 
les ensembles \(\left\{ u_0 ,\dots,u_n \right\}\)
tels que 
\[
\conv\left\{ u_0 ,\dots,u_n \right\} \in K.
\]
\end{rmk}

\begin{thm}[Realization geométrique     ]
Soit  \(A\) un complexe simplicial 
abstrait de dimension \(n\).
Alors il existe un complexe simplicial geométrique 
\(K \subset \R^{2d+1}\) tel que \(A(K) = A\).
\end{thm}

\red{i'd like an extension to paste an image from the clipboard directly in vscode, in images folder, and adds the
latex image snipet}


\begin{ex}
Le complexe
$$A=\{[0,1],[0,3],[0,5],[4,1],[2,3],[4,5],[2,1],[4,3],[2,5]\}$$
n'est pas dans \(\R^2\) mais dans \(\R^3\).
En effet, solution par le théorème de Kuratowsi.
Il s'agit d'un graphe que n'est pas planar.
\end{ex}

\begin{ex}
mais il y a aussi de graphes planars que ne se peuvent 
metre dans \(\R^2\) nonplus.
\[
A = \left\{ [0,1],[1,2],[0,2],[0,3],[1,3],[2,3] \right\}
\]
this one was the example at the first Lecture.
\end{ex}



\section{*Topologies, simplexes singuliers, et théorie des catégories}
\red{Lecture 2026-01-20}
remplacement.
this is like an extra Lecture.

\begin{df}
Une categorie est 
\end{df}

\section{Groupe d'homologie}
\red{Lecture 2026-01-22}
see my notes,
check Elementary Applied topology, Ghrist, page 30


Tout au long de cette section, \(\mathbb{F}\) désigne un corps.
On travaille avec \(\mathbb{F} = \Z_2 \).

\begin{df}
Une complexe de chaîne 
est une paire  \(\left( C_\bullet, \partial_\bullet \right)\),
où \(C_\bullet = (C_i)_{i\in \Z}\) est une famille de espaces vectoriels sur \(\F\)
et \(\partial_\bullet = (\partial_i\colon C_i\to C_{i-1})_{i\in \Z}\) est une famille \red{:/}  de applications linéaires
\[
\cdots \to C_i \xrightarrow{\partial_i} C_{i-1} \xrightarrow{\partial_{i-1}} C_{i-2} \to \cdots
\]
tel que 
\[
\partial_{i-1}\circ \partial_{i} = 0
\]
pour chaque \(i\in\Z\).
On dit que l'espace \(C_i\)  est de degré \(i\)
et que \(\partial_i \) est le differetiel \red{confirm this terminology} 
de degré \(i\).
\end{df}

Comme d'habitude, on omit the endowing stuff, et on denote le complexe de chaîne
simplement par \(C_\bullet\),
or even par \(C\) seulement.

\begin{rmk}\hfill
\begin{itemize}
\item On se souvient toujours que le dubindex de \(\partial\) est le index de son domaine.
\item On remarque que la condition \(\partial_{i-1}\circ \partial_{i}\) est équivalente à \(\im \partial_i \subseteq \Ker \partial_{i-1} \).
\item On note \(B_i(C_\bullet) = \im \partial_i\) et \(Z_i(C_\bullet) = \Ker \partial_i\) pour chaque \(i\in \Z\).
\item On oublie le subindex de \(\partial_i\). Donc on écrit \(\partial^2 = 0\).
\end{itemize}

\end{rmk}

\red{quelle est la difference avec complexe differentiel?}


\begin{df}

Le i-ème groupe d'homologie d'un complexe simplicial \(C_\bullet\) est 
\[
H_i(C_\bullet) = \Ker \partial_i / \im \partial_{i+1}.
\]
\end{df}



On remarque que \(H_i\) est un espace vectoriel,
éntant que qoutient des espaces vetoriels.

Avec notre notation de antes, 
on a \(H_i(C_\bullet) = Z_i/ B_i\).

\subsection{``but d'ajourdhui'}

on a vu les définitions précedentes sans réference a ce on avait étudie avant.
Maintenant, on fait la connection.
Étant donné un complexe simplicial abstrait \(A\),
on voudrait  construire un complexe de chaînes 
\(C(A)\).
même si  on donne un complexe simplices geométrique.


Soit \(A \) un complexe simplicial abstrait.
Pour chaque \(i\in \Z\),
on pose 
\(C_i\) l'espace vectoriel d ebase l'ensemble de i-simplexes dans \(A\). (i.e. de dim i)

Alors 
\begin{align*}
C_0 &= \langle\,  \mathrm{points} \, \rangle\\
C_1 &= \langle\,  \mathrm{arêtes} \, \rangle\\
C_2 &= \langle\,  \mathrm{triangles} \, \rangle\\
    &\ \, \vdots                           \\
C_i &= \langle\,  i\mathrm{-simplexes} \, \rangle
\end{align*}

On remarque que \(C_i = 0\), l'espace vectoriel nul, pour Tout \(i \in \Z_{\leq 1}\).
Alors on pose \(C(A) = (C_i)_{i\in\Z}\).

On construit Maintenant les applications linéaires.

\[
\partial_1 \colon C_1 \to C_0, \quad ab\mapsto a+b
\]

% \(\cdot - \cdot\mapsto \) 
\red{définir \(\partial_i\)}

Si \(A\) est un complexe simplicial abstrait,
on a le \(i\)-ème groupe d'homologie 
\[
H_i(A) = H_i(C(A)).
\]
on note \(\beta_i = \dim H_i(A)\).
La dimension étant que espace vectoriel,
laquel est égal a dim ker - dim img, 
par un théorème d'algèbre linéaire.


\begin{ex}
Soit 
\[
A = \left\{ 
        [5,6], [1,2,3], [1,4], [3,4]
 \right\}   
\]

\begin{figure}[H]
    \centering
    \incsvg{1}{cs-un-dim-trou-un-arete}
    %\caption{}
    %\label{fig:cs-un-dim-trou-un-arete}
\end{figure}
\end{ex}

\subsection{Complexe de Cech}

\section{2026-01-29}

\begin{df}
Une suite exacte corte d'espaces vectoriels
est 
\end{df}

on peut le voir comme un cas particulier d'un complexe de chaine
ou \(C_i = 0\) pour tou \(i<0, i>2\).

\red{convention: when we say the diagram commutes, we mean the solid lines. the dashed arros are about the claim of the thm}

\begin{lm}
Donné un diagramme commutatif de lignes exactes 
\[
\begin{tikzcd}
0 \arrow[r] & X \arrow[d, "u"] \arrow[r, "f"] & Y \arrow[r, "g"] \arrow[d, "v"] & Z \arrow[d, "w", dashed] \arrow[r] & 0 \\
0 \arrow[r] & X' \arrow[r, "f'"]              & Y' \arrow[r, "g'"]              & Z' \arrow[r]                       & 0
\end{tikzcd}
\]
Il existe une fonction linéaire \(w\colon Z\to Z' \)
tel que le carré à droite commute,
i.e. \(w\circ g = g' \circ v\).
\end{lm}

\red{maybe the zeros to the left are not needed.}

\begin{proof}
Soit \(z\in Z\).
\end{proof}


\begin{df}
Une filtration d'un complexe simplicial \(K\) 
est une suite de complexes simpliciaux
\[
K(0)\subseteq K(1) \subseteq \cdots \subseteq K(n) = K
\] 
\end{df}


\red{exemple: the big grid}