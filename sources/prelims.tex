\chapter{Des choses de base}


\red{Lecture 2026-01-08}

may be check Schenck Chapter 4
Homology I: Simplicial Complexes
to Sensor Networks

and 
chap 3 of 
Edelsbrunner

\section{Motivation}

Soit   \(K\) un espace topologique.

\subsection{Les groupes d'homologie}

On voudrait compter les trous d'un espace topologique quelconque.
Par exemple, 

\begin{itemize}
\item la ligne réel n'a pas de trous
\item si on prendre un segment de recte et join the bouts, on obtient un circle. il n'y a pas de trou dans le borde (otherwise it is not a circle). mais il enclose un trou.
\item 
\item la sphère n'a pas de trous dans la surface (otgerwise is not an sphère), mais elle a un trou dedans, mais ce different de le trou que le circle enclose
\item la sphère moins un disque a deux trous. pas vraiment. 
\item la figure du \texttt{8} a deux trous.i
\end{itemize}


\begin{figure}[H]
    \centering
    \incsvg{1}{efwjbjkwf}\quad
    \incsvg{1}{sphere}\quad
    \incsvg{1}{sphere-trou}
    %\caption{}
    %\label{fig:efwjbjkwf}
\end{figure}

\(K_1\) est la ligne seulement, pas l'air qu'elle contient.
Sino, on aurait peint l'interiéur.


Pour illustrer, on note par 
\(H_0\) le nombre de composantes conexes,
\(H_1\) le nombre de trous de dimension 1, i.e., trous planars,
\(H_2\) le nombre de trous bidimensionals, etc.

Alors dans les figures ci-dessus, on a 
\begin{alignat*}{3}
H_0(K_1) &= \Z,\quad  && H_1(K_1) = \Z^4,\quad  && H_2(K_1) = 0\\
H_0(K_2) &= \Z,\quad  && H_1(K_2) = \Z,\quad  && H_2(K_2) =\Z\\
H_0(K_3) &= \Z,\quad  && H_1(K_3) = 0,\quad  && H_2(K_3) = 0.
\end{alignat*}

We ``encode'' the number of conected components in the exponent.
Notice that we always have \(H_0\geq 1\).


On va voir deux théorèmes importants :

\begin{thm}
Soient \(K\) et \(K'\) deux espaces topologiques.
Si \(K\cong K'\), 
alors \(H_p(K)\cong H_p(K')\).
\end{thm}

On remarque que le premier \(\cong\) est un isomorphisme d'espaces topologiques 
et le deuxième \(\cong\) est un isomorphisme de groupes.


\begin{thm}
Tout espace topologique est isomorphe à un complexe simplicial.
\red{est-ce que ceci est vrai?}
\end{thm}

BUT : Définir \emph{complexe simplicial}.


\begin{df}
Un \(n\)-simplex \(\sigma\) est 
\end{df}

\begin{df}
On dit que \(n\) points \(u_1, \dots, u_n\in \R^d\)
sont \textbf{linéairement independants}
si 
$$
\dim \spn \left\{ u_1, \dots, u_n \right\} = n
$$
\end{df}

Il n'est pas necesaire d'ajouter la condition \(n\leqslant d\)
dans la définition.

\red{
 Is a Singleton affinely independent? By definition, no. But it should!?
* Affine Independence is interesting from R² onwards, not R¹. Nest pas?
}

\begin{df}
On dit que \(n+1\) points \(u_0, \dots, u_n\in \R^d\)
sont \textbf{affinement independants}
si les \(n\) points \(u_0 - u_1, \dots, u_0 - u_n\)
sont linéairement independants
\end{df}

Le point \(u_0\) n'est pas especial.
On peut chosir un autre point de notre liste, disons \(u_j\),
er verifier que 
\(u_j - u_1, \dots, u_j - u_n\)
est linéairement independant.

Pour illustrer l'idée d'être affinement indépendants,
on considere trois points.
Si on prend \(u_0\) comme origen,
alors on a deux vecteurs, et on checke si son colineaires ou pas.
les points aff. ind. si il ne sont pas dans la même 
ligne.
l'origin n'importe pas, 
il s'agit d'une question geometrique 
et pas de la position dans le plan.

\begin{figure}[H]
    \centering
    \incsvg{1}{idde-affinement-indep}
    %\caption{}
    %\label{fig:idde-affinement-indep}
\end{figure}

\begin{cis}
Soit \(\{u_0, \dots, u_n\}\) linéairement independant.
Quel point peut-on ajouter so that 
on obtient un esemble affinement independant ?
\end{cis}


La combinaison convexe de   \(u_1, \dots, u_n\in \R^d\) \red{they need not be aff.indep.} affinement independants
est 
\red{give intuition}
\[
\conv\{u_0, \dots, u_n\}
=
\left\{ 
\sum_{i=0}^{n}\lambda_iu_i \ \bigg|\  \lambda_i\geq 0 \text{ et } \lambda_1+ \cdots + \lambda_n =1
\right\}
\]

\begin{df}
Un  \(n\)-\textbf{simplex}
est la combinaison convexe de d'un ensemble fini de points affinement independants.
Si \(\sigma = \conv\{u_0, \dots, u_n\}\)
est un simplex, 
son dimension est \(\dim \sigma = n\).
\end{df}

\begin{rmk}
deux simplex peuvent avoir la même dimension 
même si apartient a differents espaces.
E.g. un triangle dans le plan, un triangle dans l'espace, dans R4
il a toujours dimension 3.
\end{rmk}

\red{define standard simplexes?}
there are many segments, triangles and tetahedra, mais 
il y quelconques que sont distinguished.

\begin{ex}
\begin{enumerate}[label=(\roman*)]
\item Pour \(n=0\), on a \(\sigma = \conv\{u_0\} = \{u_0\}\), un singleton.
\item segment
\item triangle
\item tetraèdre
\end{enumerate}
\end{ex}

Soit \(\sigma = \conv\{ u_0, \dots, u_n \}\)
un \(n\)-simplex 
et 
\(S\subset \{ u_0, \dots, u_n \}\) une partie non vide.
Alors \(\tau = \conv S\) est un simplex de dimension \(|S| - 1 \).
On dit que \(\tau\) est une face de \(\sigma\).
On écrit \(\tau\leq \sigma\).

\begin{rmk}
On dit  que \(\tau\) est \textbf{propre} si \(\tau\neq \sigma\).
Dans ce cas là on note \(\tau < \sigma\).
\end{rmk}

\begin{ex}
Soient \(u_0, \dots, u_3\) affinement independants.
Alors \(\sigma = \conv\{u_0, \dots, u_4\}\)
est un tetraèdre
et \(\sigma = \conv\{u_0, \dots, u_3\}\) 
est un triangle.
\red{put a picture of a tetraèdre with a face shaded of some color}
en fait, est une face propre.
\end{ex}

Pour un simplexe \(\sigma=\conv\{u_0, \dots, u_n\}\),
l'envelope de \(\sigma\)
est 
\[
\bd \sigma = \bigcup_{\mathrm{face\ propre\ }\tau\leq \sigma} \tau.
\]
Ce signifie que l'envelope est la ligne exterieur, the boundary.
L'interiéur de \(\sigma\) est \(\sigma\setminus \bd \sigma\).
Un point appartient à l'interieur si tous les \(\lambda_i\) sont positifs.

On arrive à notre but de cette section.

\begin{df}
Un \textbf{complexe simplicial} 
\(K\) est un ensemble fini de 
simplexes tels que 
\red{but what if they come from different ambient spaces? 
we should mention this!}
\begin{enumerate}[label=(\roman*)]
\item si \(\sigma\in K\) et \(\tau\leq \sigma\), alors \(\tau\in K\) ;
\item si \(\sigma,\sigma'\in K\) et \(\sigma\cap \sigma'\neq \varnothing\), alors \(\sigma\cap \sigma' \leq \sigma  \).
\end{enumerate}
La \textbf{dimension} de \(K\) est \[
\dim K = \max_{\sigma\in K} \left( \dim \sigma \right).
\]
L'espace associé \red{underlying space} est 
\[
|K| = \bigcup_{\sigma\in K} \sigma.
\]
C'est un espace topologique avec la topologie
\red{inherited from the ambient Euclidean space in which the simplices live}
so what does this mean?
\end{df}

\begin{ex}
On considère le complexe simplicial
\[
K= \left\{ 
    \conv\{u_0, u_1,u_2\}, 
    \conv\{u_0, u_1\},  
    \conv\{u_0, u_2\},  
    \conv\{u_1, u_2\}, 
    \conv\{u_2, u_3\}, 
    u_0, u_1,u_2, u_3.
 \right\}
\]
\red{shouldn't be the singletons instead of the actual points in there?}
Graphiquement : 
\begin{figure}[H]
    \centering
    \incsvg{1}{ex-simplex-tri}\\
    %\caption{}
    %\label{fig:ex-simplex-tri}
\end{figure}
\end{ex}

\red{pintar los dibujos le quita formalidad?}

\begin{ex}
Soit \(K\) le complexe simplicial du dernier exemple.
Si on ajoute le segment \(\conv\{u_1,u_3\}\),
on obtient le complexe suivant
\begin{figure}[H]
    \centering
    \incsvg{1}{ex-simplex-quat}\\
    %\caption{}
    %\label{fig:ex-simplex-quat}
\end{figure}
\end{ex}

et on peut avoir des choses qui ne sont pas connectées.

\begin{ex}[Non exemple]
Le problème ce que l'intersection 
des segment 
\(\conv\{u_1,  u_2\}\)
et 
\(\conv\{u_3,  u_4\}\)
n'est pas un element du \(K\).
Mais on peut bien l'ajouter.
\begin{figure}[H]
    \centering
    \incsvg{1}{ex-non-complexe-simplicial}\\
    %\caption{}
    %\label{fig:ex-non-complexe-simplicial}
\end{figure}
\end{ex}

\red{i like this quote from Edelsbrunner}
\red{It is often easier to construct a complex ab-
stractly and worry about how to put it into Euclidean space later, if at all.}


\begin{df}
Un \textbf{complexe simplicial abstrait} \(A\)
est un ensemble fini des ensembles finis, sauf \(\varnothing\),
tels que 
si \(\sigma\in A\) et \(\tau\subset \sigma\),
alors 
\(\tau\in A\). \red{this means \(A\) is closed below, sth like that}
La dimension de \(A\) est 
\[
\dim A = \max_{\sigma\in A} \left\{ |\sigma| - 1 \right\}.
\]
\end{df}

\begin{ex}
L'ensemble
\[
A = 
\left\{ 
    \left\{ 0,1,2 \right\},
    \left\{ 0,1 \right\},
    \left\{ 0,2 \right\},
    \left\{ 1,2 \right\},
    \left\{ 2,3 \right\},
    \left\{ 0 \right\},
    \left\{ 1 \right\},
    \left\{ 2 \right\},
    \left\{ 3 \right\}
\right\}
\]
\end{ex}

Pour un ensemble fini de points \(A = \left\{ a_0, \dots, a_n \right\}\),
on note 
par \([a_0, \dots, a_n]\)
le ensemble de tous parties 
nonvides de \(A\).
Équivalentement, \red{?}
\[
[a_0, \dots, a_n] = \mathscr{P}(\left\{ a_0, \dots, a_n \right\}) \setminus \left\{ \varnothing \right\}.
\]



\begin{ex}
On a 
\[
[0,1,2] = 
\left\{ 
    \left\{ 0,1,2 \right\},
    \left\{ 0,1 \right\},
    \left\{ 0,2 \right\},
    \left\{ 1,2 \right\},
    \left\{ 0 \right\},
    \left\{ 1 \right\},
    \left\{ 2 \right\},
    \left\{ 3 \right\}
\right\}.
\]
\end{ex}


\begin{rmk}
Si \(K\) est un complexe simplicial geométrique,
son complexe abstrait \(A(K)\) est donné par 
les ensembles \(\left\{ u_0 ,\dots,u_n \right\}\)
tels que 
\[
\conv\left\{ u_0 ,\dots,u_n \right\} \in K.
\]
\end{rmk}

\begin{thm}[Realization geométrique     ]
Soit  \(A\) un complexe simplicial 
abstrait de dimension \(n\).
Alors il existe un complexe simplicial geométrique 
\(K \subset \R^{2d+1}\) tel que \(A(K) = A\).
\end{thm}

\red{i'd like an extension to paste an image from the clipboard directly in vscode, in images folder, and adds the
latex image snipet}